\chapter*{\centering Preface}   
% Testo della prefazione
TODO.
%Heterogeneous parallel devices are becoming widely diffused in the embedded systems domain, mainly because of the opportunities to improve performance and, at the same time, other orthogonal metrics (e.g. cost, power and energy dissipation, area etc.). In such a context, the introduction of safety integrity levels (dictated by the standards) into embedded applications, considering shared resources on a heterogeneous parallel HW platform, adds further challenges in industrial and academic researches. The exploitation of virtualization technologies allows to guarantee isolation and to satisfy certification requirements, but introduces scheduling overhead and new HW/SW and SW/SW partitioning challenges. The choice of the best hypervisors partitioning model, the number of criticality levels and the number of partitions are just some of them. In this scenario, this work faces the role of Design Space Exploration for embedded systems based on heterogeneous parallel architectures and subject to mixed-criticality system requirements, while considering the exploitation of hypervisor-based SW partitions to better manage isolation. In particular, it presents an evolutionary partitioning and mapping approach integrated into a reference Electronic System Level HW/SW Co-Design framework to propose and early validate design solutions by means of experimental setups.