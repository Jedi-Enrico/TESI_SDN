\begin{abstract}
The heterogeneity of modern network infrastructures involves different devices and protocols bringing out several issues in organizing and optimizing network resources, making their coexistence a very challenging engineering problem. In this scenario, Software Defined Network (SDN) architectures decouple control and forwarding functionalities by enabling the network devices to be remotely configurable/programmable in run-time by a controller, and the underlying infrastructure to be abstracted from the application layer and the network services, with the final aim of increasing flexibility and performance. As a direct consequence identifying an accurate model of a network and forwarding devices is crucial in order to apply advanced control techniques such as Model Predictive Control (MPC) to optimize the network performance. An enabling factor in this direction is given by recent results that appropriately combine System Identification and Machine Learning techniques to obtain predictive models using historical data retrieved from a network. This paper presents a novel methodology to learn, starting from historical data and appropriately combining autoregressive exogenous(ARX) identification with Regression Trees (RT) and Random Forests (RF), an accurate model of the dynamical input-output behavior of a network device that can be directly and efficiently used to optimally and dynamically control the bandwidth of the queues of switch ports, within the SDN paradigm. Both the Mininet network emulator environment and a real dataset obtained from measurements of the network of an Italian internet service provider (Sonicatel S.r.l.). have been used to validate the prediction accuracy of the derived predictive models. The benefits of the proposed dynamic queueing control methodology in terms of Packet Losses reduction and Bandwidth savings (i.e. improvement of the Quality of Service) has been finally demonstrated.
\end{abstract}